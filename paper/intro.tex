\section{Presentation}

Contracts were incorporated into the C++26 draft. However, there are a number of
concerns that still remain unsolved. We recommend that unless those concerns are
properly address the proposal in its current form should be removed from the
C++26 draft.

Here is a summary:

\begin{itemize}

\item \textbf{Lack of sufficient experience}: The contracts feature is
a major feature that lacks of enough deployment experience. While it is true
that some portions have been experimented in projects. There are many features
that have not been tried sufficiently.

Moreover, at this point we also lack of enough user experience, implementation
experience and build system experience. The latter is specially significant in
presence of mixed mode builds. We should also have experience in multiple
domains. What is acceptable in one problem domain becomes critically
unacceptable in a different domain.

The safest path would be to get more experience by providing the feature either
in a technical specification or in a white paper, so that all the issues are
better understood.

\item \textbf{Mixed mode builds}: In P3835~\cite{p3835r0} examples are given for
a header file that contains an \cppkey{inline} function with a contract assertion and is
used from different translations units with different evaluation semantics. The
same applies to \cppkey{constexpr} functions, \cppkey{consteval} functions and
templates.

This a major safety issue as the same assertion might be checked or not
depending on the caller. Yet subsequent code might depend on the validity of
that assertion.

\item \textbf{Multiple dependent assertions}: When there are multiple dependent
assertions (a precondition that checks for null pointer and another precondition
that dereferences that pointer), the evaluation in non-terminating modes may
lead to undefined behavior.

This is a safety issue. It may be mitigated by not evaluating dependent contract
assertions when the first assertion fails.

\item \textbf{Unsafe consequences}: In paper P3829~\cite{p3829r0} examples are
given on how contract assertions may lead to situations where a critical check
may end-up elided. As the papers states this might be a powerful tool for new
supply-chain attacks.

This is another major safety issue. Contracts should not worsen the safety
dimension of the language.

\item \textbf{Constification}: Clause 6.11.1/4 makes any variable to be
\cppkey{const} within the predicate of a contract assertion. This also applies
to the \cppkey{this} pointer. This is specially problematic when invoking an
overloaded function, as the overload resolution mechanism might select a
different version than in other context.

In addition, clause 9.4.1/7 states that by a using a non-reference parameter of
a function in a post-condition the parameter magically becomes \cppkey{const}.
In this case, adding a post-condition to a function is changing the signature of
the function, which seems counter-intuitive.

This is major concern from the teachability point of view, as it will make the
code harder to understand. Moreover, this might not be acceptable for projects
where maintainability and simplicity are major drivers. It is also worth to
note that constification violates a fundamental design principle for C++: given
the same object, the same operation should have the same effect.

\item \textbf{No support for function pointers}: Clause 9.4.1/8 states (inside a
note) that ``\emph{ A pointer to function, pointer to member function, or function
type alias cannot have a function-contract-specifier-seq associated directly
with it}''. Pointer to functions are a first class citizen in C++.

Pointer to functions and pointer to member functions are widely used in many
code bases. Not supporting contracts on them will be a barrier to adopt
contracts in those code bases.

\item \textbf{No support for virtual functions}: C++ supports multiple styles of
programming, including object-oriented programming where virtual functions play
a major role. However, the contract feature explicitly forbids the use of
contracts specifiers for virtual functions (cluase 9.4.1/6). This is a major
drawback of the feature.

There are large codebases written in C++ that make extensive use of virtual
functions. By providing a solution that does not support one of the major styles
the language is doing a disservice to such portion of the user community.


\end{itemize}
